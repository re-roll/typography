\section{Příklad 2}
% Jako parametr zadejte skupinu (A-H)
\druhyZadani{B}

\begin{center}
	{Musíme najít $R_1$, takže budeme vypočitat pomocí ekvivalentního obvodu, ve kterém:}
\end{center}

\begin{align*}
	I_{R_1} = \frac{U_i}{R_i + R_1}
\end{align*}

\begin{center}
	{Nejprvé zjednodušíme obvod (s odpojeným $R_1$) pomocí Ohmova zákona}
\end{center}

\begin{align*}
	R_{45} = R_4 + R_5 = 790 \Omega \\
	R_{345} = \frac {R_3 \cdot R_{45}}{R_3 + R_{45}} = 344,2 \Omega \\
	R_i = \frac {R_2 \cdot R_{345}}{R_2 + R_{345}} = 163,1 \Omega 
\end{align*}

\begin{center}
	{Teď můžeme najít $R_{ekv}$, a potom $U_{R_2}$}
\end{center}

\begin{align*}
	R_{ekv} = R_2 + R_{345} = 654,2 \Omega \\
	I = \frac {U}{R_{ekv}} = 0,1529 A\\
	U_{R_2} = R_2 * I = 47,4 V = U_i \quad (pozn. U_{R_2} = U_i !)
\end{align*}

\begin{center}
	{Teď už můžeme použit formuli pro $I_{R_1}$ a potom najdeme $U_{R_1}$}
\end{center}

\begin{align*}
	I_{R_1} = \frac{U_i}{R_i + R_1} = 0,222 A \quad U_{R_1} = I_{R_1} \cdot R_1 = 11,1 V
\end{align*}