\documentclass[11pt, twocolumn, a4paper]{article}
\usepackage[text={18cm, 25cm}, left=1.5cm, top=2.5cm]{geometry}
\usepackage[czech]{babel}
\usepackage[utf8]{inputenc}
\usepackage[IL2]{fontenc}
\usepackage{mathtools, amssymb, amsthm}
\usepackage{times}
\usepackage[hidelinks]{hyperref}
\begin{document}

    \begin{titlepage}
        \begin{center}
            
            \textsc{\huge{Vysoké učení technické v Brně}}
            \\[0.4em]
            \textsc{\LARGE{Fakulta informačních technologií}}
            \\
                
            \vspace{\stretch{0.382}}
            
            \LARGE{Typografie a publikování \,--\, 2. projekt
            \\[0.3em]
            Sazba dokumentů a matematických výrazů}
               
            \vspace{\stretch{0.618}}
        
        \end{center}    
        \Large{2022} \hfill \Large{Dmitrii Ivanushkin (xivanu00)}
    \end{titlepage}
    
    \section*{Úvod}
    V této úloze si vyzkoušíme sazbu titulní strany, matematických vzorců, prostředí a dalších textových struktur obvyklých pro technicky zaměřené texty (například rovnice (\ref{eq:2}) nebo Definice \ref{def:1} na straně \pageref{def:1}). Pro vytvoření těchto odkazů používáme příkazy {\fontfamily{qcr}\selectfont \string\label}, {\fontfamily{qcr}\selectfont \string\ref}  a {\fontfamily{qcr}\selectfont \string\pageref.}
    
    Na titulní straně je využito sázení nadpisu podle optického středu s využitím zlatého řezu. Tento postup byl probírán na přednášce. Dále je na titulní straně použito odřádkování se zadanou relativní velikostí 0,4~em a 0,3~em.
    
    \section{Matematický text}
    Nejprve se podíváme na sázení matematických symbolů a výrazů v plynulém textu včetně sazby definic a vět s využitím balíku {\fontfamily{qcr}\selectfont amsthm}. Rovněž použijeme poznámku pod čarou s použitím příkazu {\fontfamily{qcr}\selectfont \string\footnote}. Někdy je vhodné použít konstrukci {\fontfamily{qcr}\selectfont \string$\string{\string}\string$} nebo {\fontfamily{qcr}\selectfont \string\mbox\string{\string}}, která říká, že (matematický) text nemá být zalomen. 
    \theoremstyle{definition}
    \newtheorem{definition}{Definice}
    \begin{definition} \label{def:1}
        Nedeterministický Turingův stroj \emph{(NTS) je šestice tvaru} \emph{$M = (Q, \Sigma, \Gamma, \delta, q_0, q_F)$, kde:}
    \end{definition}
    \begin{itemize}
        \item{\emph{$Q$ je konečná množina }vnitřních (řídicích) stavů,}
        \item{$\Sigma$ \emph{je konečná množina symbolů nazývaná} vstupní abeceda, $\Delta \notin \Sigma$,}
        \item{$\Gamma$ \emph{je konečná množina symbolů, $\Sigma \subset \Gamma$, $\Delta \in \Gamma$, nazývaná} pásková abeceda,}
        \item{${\delta}$ : (\emph{$Q$ $ \backslash $ ${\lbrace q_F \rbrace)\times \Gamma \rightarrow 2^{Q \times (\Gamma \cup \lbrace L, R \rbrace)}}$, kde L, R $\notin$ $\Gamma$, je parciální} přechodová funkce, \emph{a} }
        \item{\emph{$q_0$ $ \in $ $Q$ je }počáteční stav \emph{a $q_F$ $ \in $ $Q$ je }koncový stav.}
    \end{itemize}
    
    Symbol $\Delta$ značí tzv. \emph{blank} (prázdný symbol), který se vyskytuje na místech pásky, která nebyla ještě použita.
    
    \emph{Konfigurace pásky} se skládá z nekonečného řetězce, který reprezentuje obsah pásky, a pozice hlavy na tomto řetězci. Jedná se o prvek množiny $\lbrace \gamma\Delta^\omega$ $|$ $\gamma \in \Gamma^* \rbrace \times \mathbb{N}$\footnote{Pro libovolnou abecedu $\Sigma$ je ${\Sigma^\omega}$ množina všech \emph{nekonečných} řetězců nad $\Sigma$, tj. nekonečných posloupností symbolů ze $\Sigma$}
    \emph{Konfiguraci pásky} obvykle zapisujeme jako ${\Delta xyz\underline{z}x \Delta\dots}$ (podtržení značí pozici hlavy).
    \emph{Konfigurace stroje} je pak dána stavem řízení a konfigurací pásky. Formálně se jedná o prvek množiny $Q \times \lbrace \gamma\Delta^\omega$ $|$ $\gamma \in \Gamma^* \rbrace \times \mathbb{N}$
    \subsection{Podsekce obsahující definici a větu}
    \begin{definition} \label{def:2}
        Řetězec $w$ nad abecedou $\Sigma$ je přijat NTS~\emph{$M$, jestliže $M$ při aktivaci z počáteční konfigurace pásky $\underline{\Delta} w\Delta$ \dots a počátečního stavu $q_0$ může zastavit přechodem do koncového stavu $q_F$, tj. $(q_0, \Delta w, \Delta^w, 0) \mathrel{\substack{*\\\vdash\\M}} (q_F, \gamma, n)$ pro nějaké $\gamma \in \Gamma^*$ a $n \in \mathbb{N}$}
        
       \emph{ Množinu $L(M) = \lbrace w$ $|$ $w$ je přijat NTS $M \rbrace \subseteq \Sigma^*$ nazýváme} jazyk přijímaný NTS $M$.
    \end{definition} 
        Nyní si vyzkoušíme sazbu vět a důkazů opět s použitím balíku {\fontfamily{qcr}\selectfont amsthm}.
        
    \newtheorem{theorem}{Věta}
        
    \begin{theorem}
        \emph{Třída jazyků, které jsou přijímány NTS, odpovídá} rekurzivně vyčíslitelným jazykům.
    \end{theorem}
    
    \section{Rovnice}
    
    Složitější matematické formulace sázíme mimo plynulý text. Lze umístit několik výrazů na jeden řádek, ale pak je třeba tyto vhodně oddělit, například příkazem {\fontfamily{qcr}\selectfont \string\quad}.
    
    \[x^2 - \sqrt[4]{y_1 * y_2^3} \quad x > y_1 \geq y_2 \quad z_{z_z} \neq \alpha_1^{\alpha_2^{\alpha_3}}\]
    
    V rovnici (\ref{eq:1}) jsou využity tři typy závorek s různou explicitně definovanou velikostí.
    
    \begin{equation} \label{eq:1}
        x = \bigg\{ a \oplus \Bigl[ b \cdot \bigl(c \ominus d \bigr) \Bigr] \bigg\}^{4/2}
    \end{equation}
    
    \begin{equation} \label{eq:2}
        x = \lim_{\beta\to\infty} \frac{\tan^2\beta - \sin^3\beta}{\frac{1}{\frac{1}{\log_{42} x} + \frac{1}{2}}}
    \end{equation}
    
    V této větě vidíme, jak vypadá implicitní vysázení limity $\lim_{n\to\infty}f(n)$ v normálním odstavci textu. Podobně je to i s dalšími symboly jako $\bigcup_{N \in \mathcal{M}}N$ či $\sum_{j=0}^n x_j^2$ 
    S vynucením méně úsporné sazby příkazem {\fontfamily{qcr}\selectfont \string\limits} budou vzorce vysázeny v podobě $\lim\limits_{n\to\infty} f(n)$ a $\sum\limits_{j=0}^n x_j^2$.
    
    \section{Matice}
    Pro sázení matic se velmi často používá prostředí {\fontfamily{qcr}\selectfont array} a závorky ({\fontfamily{qcr}\selectfont \verb|\left|}, {\fontfamily{qcr}\selectfont \verb|\right|}). 
    
    \[ \textbf{A} = 
    \left| 
    \begin{array}{cccc}
        a_{11} & a_{12} & \dots & a_{1n} \\
        a_{21} & a_{22} & \dots & a_{2n} \\
        \vdots & \vdots & \ddots & \vdots \\
        a_{m1} & a_{m2} & \dots & a_{mn}
    \end{array} 
    \right| 
    = \left|
    \begin{array}{cc}
        t & u \\
        v & w
    \end{array}
    \right| 
    = tw - uv\]
    
    Prostředí {\fontfamily{qcr}\selectfont array} lze úspěšně využít i jinde.
    
    \[ \binom{n}{k} = 
    \left\{ 
    \begin{array}{cl}
        \frac{n!}{k!(n-k)!} & \mbox{pro $0 \leq k \leq n$} \\
        0 & \mbox{pro $k > n$ nebo $k < 0$}
    \end{array} 
    \right. \]

\end{document}
